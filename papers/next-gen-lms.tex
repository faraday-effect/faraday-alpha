\documentclass{article}

\usepackage[colorlinks]{hyperref}
\usepackage{amssymb}

\newcommand{\vcalc}{\textsc{VisiCalc}}

\title{Humanizing Learning Technology}
%\title{Humanizing Learning Technology\\with Actionable Educational Intelligence}
\author{Tom Nurkkala, PhD\\\texttt{tom.nurkkala@gmail.com}}

\begin{document}
\maketitle

\begin{abstract}
  Next generation learning management systems
  should provide actionable educational intelligence
  that supports continuous, measurable improvement
  in teacher productivity and student outcomes.
  Such systems
  should re-personalize the relationship
  between teacher and student
  into a modern apprenticeship model
  that encourages mastery learning
  without sacrificing scalability
  to larger populations of students.
  Importantly, these systems promise to
  address important and timely concerns
  related to distance learning
  both during the pandemic and beyond.
\end{abstract}

\section{Limitations of Current Systems}
\label{sec:limitations}

Current and emerging learning management systems
represent a considerable improvement over the paper grade book,
but suffer two key limitations.
First,
they are \emph{not actionable}:
they provide little actionable information
that can be employed directly to engage students,
enhance learning outcomes,
and simplify course management.
Second,
they are \emph{not personalized}:
they emphasize scalability
at the expense of personalized interaction
between student and teacher.

\subsection{Not Actionable}
\label{sec:not-actionable}

Prior to the 1979 introduction
of \href{https://en.wikipedia.org/wiki/VisiCalc}{\vcalc{}},
the first computerized spreadsheet program,
business people used pencils and paper spreadsheets,
pre-printed with the now-universally-familiar rows and columns of cells.
Although \vcalc{} and its offspring
represented a huge advance over paper,
even modern spreadsheet software does little more
than automate the manual calculations
required by its paper progenitors.

By the 1990's, business software had gone beyond the spreadsheet,
embracing an important class of sophisticated systems
known collectively as Business Intelligence (BI) software.
While electronic spreadsheets
remain indispensable tools for organizing raw data,
BI systems transform data into actionable information
that can be used by business people
to analyze, manage, and empower their organizations.

In educational technology,
the computerized learning management system (LMS)
represents an advance over the paper grade book
that is analogous to \vcalc{}'s superiority over the paper spreadsheet.
In addition to tracking scores and grades,
systems like
\href{https://www.blackboard.com/}{Blackboard},
\href{https://moodle.org/}{Moodle},
and
\href{https://www.instructure.com/canvas/}{Canvas}
are adept at distributing content,
tracking homework submissions,
mediating on-line ``conversations,''
and otherwise modernizing the learning environment.
But also like the electronic spreadsheet,
the current generation of LMS
does not go much beyond organizing raw data.
In particular,
the present-day LMS
does not provide the student or the teacher
with significant actionable information
that helps analyze, manage, or empower the learning enterprise.

\subsection{Not Personalized}
\label{sec:not-personalized}

Recent educational technologies
have exhibited breathless enthusiasm
for scaling education to the very large.
Distance learning,
on-line degree programs,
and the ominously named Massive Open Online Course (MOOC)
make this tendency evident.
Examples abound from non-profit initiatives like
\href{https://www.edx.org/}{EdX} and
\href{https://www.khanacademy.org/}{Khan Academy}
and from for-profit companies like
\href{https://www.coursera.org/}{Coursera}
and
\href{https://www.udacity.com/}{Udacity},
not to mention the central place of technology
in the business models of for-profit institutions
like
\href{https://waldenu.edu/}{Walden} and the
\href{https://www.phoenix.edu/}{University of Phoenix}.

The emphasis on massively scalable educational opportunity
is understandable in an age of financial challenges in education,
large numbers of non-tradi\-tional students,
ubiquitous mobile technology,
global Internet access,
and billion-member online social networks.
But in the headlong rush
to deliver pre-packaged content to the masses,
we have neglected the historically significant relationship
between teacher and student
that has been the hallmark of deep learning since Socrates himself.
Significant, impactful, and long-lasting educational outcomes
are surely more likely to spring
from an individualized mentoring relationship
than from a generic video presentation or on-line chat room.

\section{Vision for a New System}
\label{sec:vision}

As the antidote to the
limitations enumerated in Section~\ref{sec:limitations},
the next generation of learning management system
should exhibit these interlocking characteristics:
\begin{enumerate}
\item \emph{Actionable}.
  It should distill raw learning data
  into educational intelligence
  on which students and (especially) teachers can act.
\item \emph{Personalized}.
  It should make explicit
  the learning needs of individual students,
  informing teachers in real time
  of the need for remediation
  or the opportunity for enhanced learning.
\item \emph{Scalable}.
  It should scale
  to a reasonably large numbers of students
  without unnecessarily burdening the teacher
  or compromising the foregoing characteristics.
\end{enumerate}

As a unifying motif
for the next generation of learning management system,
I propose a time-honored system of mastery learning: \emph{apprenticeship}.
In the medieval guild system,
a teenager was apprenticed to a master craftsman
to gain knowledge and develop skills
through direct exposure, extensive practice,
constant vigilance, and immediate feedback.
Apprenticeship was the essence of personalized education then,
and remains a viable model for effective education today.
This is not only true of vocational and technical schools.
Indeed, many undergraduate programs
require industry practica (a kind of ``apprenticeship light''),
and graduate education is itself
the clearest example of apprenticeship
in modern higher education.

Apprenticeship is highly personalized,
but its one-to-one ``faculty-student ratio''
appears to make it a hopelessly impractical anachronism.
Enter ``LMS~2.0,''
the next generation learning management system,
as the key enabling technology.
It can relieve teachers
of the mundane and time consuming details of course management.
It can provide actionable educational intelligence to teachers,
encouraging timely, strategic, and personal connection with students.
And it can scale to a large number of students.

Consider the following capabilities such a system could deliver.

\paragraph{Course Management}
\begin{enumerate}
\item Rich tools for course planning and development,
  allowing topics and activities
  to be aligned explicitly
  with course learning outcomes
  and categorized according to Bloom's taxonomy.
\item Decoupling of course planning from course scheduling, allowing existing
  courses to be scheduled in future semesters with ease.
\item Automated production of course syllabi, based on course planning and
  scheduling information.
\item
  Topical overview prepared automatically
  for each class session,
  showing students the day's topics,
  preparatory readings,
  follow-up assignments,
  upcoming exams, etc.
  Daily topics tied explicitly
  to course learning outcomes
  to contextualize the day's learning.
\end{enumerate}

\paragraph{Classroom Experience}
\begin{enumerate}
\item Teacher and students make full use of mobile technology in the classroom,
  coordinated by interaction with a central server and database.
\item Automatic attendance taken when students ``log in to class.''
\item Speaker console decoupled from classroom display,
  encouraging meaningful classroom visuals,
  avoiding ``death by PowerPoint,''
  and discouraging the
  teacher's use of a slide deck as a teleprompter.
\item Ubiquitous use of mobile devices for classroom interaction; for example,
  in-class quizzing (``clickers on steroids''), mobile-based exercises by
  individual students or small student groups, real time monitoring by teacher
  of any student's work in progress, display of selected student work for
  classroom display and discussion.
\item Efficacy of teaching strategies measured empirically using in-class
  quizzing.
\item Student control of classroom graphics on personal mobile device, allowing
  review of material without disrupting class and consuming material at an
  individualized pace.
\item Server-based tracking of the pace of student engagement
  with course material throughout a class.
\item Student submission of private comments, requests, suggestions,
  or reminders to the teacher for attention either during or after class.
\end{enumerate}

\paragraph{Individualized Mentoring}
\begin{enumerate}
\item Invitations to office hours (or other remediation)
  for students who are lagging behind the class
  (e.g., as measured by whether they ``keep up''
  when viewing classroom graphics
  as they are covered by the teacher).
\item Remedial exercises for students
  struggling with in-class exercises or quizzes.
\item Automated relief from assigned homework for students demonstrating
  excellent performance on in-class exercises.
\item Referral to campus academic enrichment programs for students
  demonstrating possible learning challenges.
\end{enumerate}

\paragraph{Educational Analytics}
\begin{enumerate}
\item Course dashboard for teachers
  that distills raw data on attendance,
  in-class quiz results,
  students in need of personal attention, etc.
\item Graphical time line
  showing which students are ``keeping up''
  with in-class material,
  which students may need more attention,
  which are asleep during class, etc.
\item
  Comparative statistical analysis
  and graphical display
  of the efficacy of various learning activities
  in the classroom.
\item
  Support for
  quantitative ``A/B testing''
  of competing strategies
  for pedagogy and student evaluation.
\item
  Analysis of time spent on each topic,
  tied to course planning information
  to help ensure class is on schedule
  and that schedules are realistic.
\item
  Information to help optimize the course
  for future semesters
  (e.g., content, pacing, student interaction, exercise relevance, etc.).
\end{enumerate}

\paragraph{Institutional Impact}
\begin{enumerate}
\item
  Opportunity for interdepartmental collaboration
  to explore the efficacy
  of learning management techniques
  in diverse disciplines.
\item
  Share results with the scholarly community
  through publication of empirical analyses
  derived from use of the system in classrooms.
\item 
  Share the software itself
  under an open-source license
  to facilitate its use by other institutions
  and its extension
  by open-source collaborators.
\item
  Plan department-wide learning outcomes
  and map outcomes to topics of specific courses.
  Ease audits by accreditation bodies.
\item
  Emergence of a global learning community
  focused on continuous improvement in teaching and learning
  through the judicious and sustainable application
  of actionable, personalized learning management technology.
\end{enumerate}

None of these capabilities
require the development of radically new technologies.
The principal \emph{challenge}
is to assemble \emph{existing} technologies
in ways that facilitate this actionable, personalized, and scalable vision
for teaching and learning.
The principal \emph{need} that must be met to make this vision a reality
is adequate participation and adequate funding
for a collaborative team with expertise in
teaching, curriculum development,
computer science, software engineering,
user experience design,
and information technology.

\section{Distance Learning and Pandemic}
\label{sec:pandemic}

The ideas outlined in Section~\ref{sec:vision}
were conceived with relation to
face-to-face classroom teaching and learning.
Due to the global pandemic,
many educational institutions
have turned to remote learning
or to hybrid models
that mix in-person and remote instruction.

In addition to challenges posed by limited infrastructure
and resources along the digital divide,
a key problem with remote or hybrid learning
is the lack of interpersonal connection
between teachers and students.
The LMS~2.0 model proposed here
offers the promise of ameliorating
this problem.
By surfacing
actionable information on student performance,
teachers will be better able
to monitor and mentor students,
even when they are learning remotely.

The prevailing sentiment
among academic stakeholders
is a desire to return to the ``old normal.''
As the pandemic ebbs,
the academy will undoubtedly move
in this direction.
That said,
there will be stakeholders
who have found the
pivot to remote and hybrid instruction
to have advantages over the old normal.
Thus,
both for the foreseeable future
(shadowed by the pandemic)
and the longer term
(after the pandemic's eradication),
the tools and technologies
enumerated here
promise to be of
considerable and continuing
utility.

\hfill\square

% \vfill
% \begin{flushleft}
%   \small
%   Revision History
%   \begin{itemize}
%   \item 2017-10-01---Initial revision
%   \item 2017-11-01---Sabbatical application submission
%   \item 2020-10-06---Revised for general distribution
%   \end{itemize}
% \end{flushleft}

\end{document}

%%% Local Variables:
%%% mode: latex
%%% TeX-master: t
%%% End:

% LocalWords:  pre EdX Coursera Udacity tradi tional
